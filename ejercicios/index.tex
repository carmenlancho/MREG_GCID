% Options for packages loaded elsewhere
\PassOptionsToPackage{unicode}{hyperref}
\PassOptionsToPackage{hyphens}{url}
\PassOptionsToPackage{dvipsnames,svgnames,x11names}{xcolor}
%
\documentclass[
  letterpaper,
  DIV=11,
  numbers=noendperiod]{scrreprt}

\usepackage{amsmath,amssymb}
\usepackage{iftex}
\ifPDFTeX
  \usepackage[T1]{fontenc}
  \usepackage[utf8]{inputenc}
  \usepackage{textcomp} % provide euro and other symbols
\else % if luatex or xetex
  \usepackage{unicode-math}
  \defaultfontfeatures{Scale=MatchLowercase}
  \defaultfontfeatures[\rmfamily]{Ligatures=TeX,Scale=1}
\fi
\usepackage{lmodern}
\ifPDFTeX\else  
    % xetex/luatex font selection
\fi
% Use upquote if available, for straight quotes in verbatim environments
\IfFileExists{upquote.sty}{\usepackage{upquote}}{}
\IfFileExists{microtype.sty}{% use microtype if available
  \usepackage[]{microtype}
  \UseMicrotypeSet[protrusion]{basicmath} % disable protrusion for tt fonts
}{}
\makeatletter
\@ifundefined{KOMAClassName}{% if non-KOMA class
  \IfFileExists{parskip.sty}{%
    \usepackage{parskip}
  }{% else
    \setlength{\parindent}{0pt}
    \setlength{\parskip}{6pt plus 2pt minus 1pt}}
}{% if KOMA class
  \KOMAoptions{parskip=half}}
\makeatother
\usepackage{xcolor}
\setlength{\emergencystretch}{3em} % prevent overfull lines
\setcounter{secnumdepth}{-\maxdimen} % remove section numbering
% Make \paragraph and \subparagraph free-standing
\makeatletter
\ifx\paragraph\undefined\else
  \let\oldparagraph\paragraph
  \renewcommand{\paragraph}{
    \@ifstar
      \xxxParagraphStar
      \xxxParagraphNoStar
  }
  \newcommand{\xxxParagraphStar}[1]{\oldparagraph*{#1}\mbox{}}
  \newcommand{\xxxParagraphNoStar}[1]{\oldparagraph{#1}\mbox{}}
\fi
\ifx\subparagraph\undefined\else
  \let\oldsubparagraph\subparagraph
  \renewcommand{\subparagraph}{
    \@ifstar
      \xxxSubParagraphStar
      \xxxSubParagraphNoStar
  }
  \newcommand{\xxxSubParagraphStar}[1]{\oldsubparagraph*{#1}\mbox{}}
  \newcommand{\xxxSubParagraphNoStar}[1]{\oldsubparagraph{#1}\mbox{}}
\fi
\makeatother


\providecommand{\tightlist}{%
  \setlength{\itemsep}{0pt}\setlength{\parskip}{0pt}}\usepackage{longtable,booktabs,array}
\usepackage{calc} % for calculating minipage widths
% Correct order of tables after \paragraph or \subparagraph
\usepackage{etoolbox}
\makeatletter
\patchcmd\longtable{\par}{\if@noskipsec\mbox{}\fi\par}{}{}
\makeatother
% Allow footnotes in longtable head/foot
\IfFileExists{footnotehyper.sty}{\usepackage{footnotehyper}}{\usepackage{footnote}}
\makesavenoteenv{longtable}
\usepackage{graphicx}
\makeatletter
\newsavebox\pandoc@box
\newcommand*\pandocbounded[1]{% scales image to fit in text height/width
  \sbox\pandoc@box{#1}%
  \Gscale@div\@tempa{\textheight}{\dimexpr\ht\pandoc@box+\dp\pandoc@box\relax}%
  \Gscale@div\@tempb{\linewidth}{\wd\pandoc@box}%
  \ifdim\@tempb\p@<\@tempa\p@\let\@tempa\@tempb\fi% select the smaller of both
  \ifdim\@tempa\p@<\p@\scalebox{\@tempa}{\usebox\pandoc@box}%
  \else\usebox{\pandoc@box}%
  \fi%
}
% Set default figure placement to htbp
\def\fps@figure{htbp}
\makeatother

\KOMAoption{captions}{tableheading}
\makeatletter
\@ifpackageloaded{bookmark}{}{\usepackage{bookmark}}
\makeatother
\makeatletter
\@ifpackageloaded{caption}{}{\usepackage{caption}}
\AtBeginDocument{%
\ifdefined\contentsname
  \renewcommand*\contentsname{Table of contents}
\else
  \newcommand\contentsname{Table of contents}
\fi
\ifdefined\listfigurename
  \renewcommand*\listfigurename{List of Figures}
\else
  \newcommand\listfigurename{List of Figures}
\fi
\ifdefined\listtablename
  \renewcommand*\listtablename{List of Tables}
\else
  \newcommand\listtablename{List of Tables}
\fi
\ifdefined\figurename
  \renewcommand*\figurename{Figure}
\else
  \newcommand\figurename{Figure}
\fi
\ifdefined\tablename
  \renewcommand*\tablename{Table}
\else
  \newcommand\tablename{Table}
\fi
}
\@ifpackageloaded{float}{}{\usepackage{float}}
\floatstyle{ruled}
\@ifundefined{c@chapter}{\newfloat{codelisting}{h}{lop}}{\newfloat{codelisting}{h}{lop}[chapter]}
\floatname{codelisting}{Listing}
\newcommand*\listoflistings{\listof{codelisting}{List of Listings}}
\makeatother
\makeatletter
\makeatother
\makeatletter
\@ifpackageloaded{caption}{}{\usepackage{caption}}
\@ifpackageloaded{subcaption}{}{\usepackage{subcaption}}
\makeatother

\usepackage{bookmark}

\IfFileExists{xurl.sty}{\usepackage{xurl}}{} % add URL line breaks if available
\urlstyle{same} % disable monospaced font for URLs
\hypersetup{
  colorlinks=true,
  linkcolor={blue},
  filecolor={Maroon},
  citecolor={Blue},
  urlcolor={Blue},
  pdfcreator={LaTeX via pandoc}}


\author{}
\date{}

\begin{document}

<div class="portada">
    <div class="portada-container">
        <!-- Header -->
        <div class="portada-header">
            <img src="../images/urjc_logo.png" alt="Logo URJC" class="portada-logo">
            
            <div class="portada-escuela">
                Escuela Técnica Superior de<br>
                Ingeniería Informática
            </div>
        </div>
        
        <!-- Contenido principal -->
        <div class="portada-main">
            <h1 class="portada-titulo">
                Métodos Estadísticos de Predicción
            </h1>
            
            <h2 class="portada-subtitulo">
                Ejercicios de la Asignatura
            </h2>
            
            <div class="portada-grado">
                Grado en Matemáticas
            </div>
            
            <div class="portada-autores-section">
                <div class="portada-autores-titulo">Autores</div>
                <div class="portada-autor">Víctor Aceña Gil</div>
                <div class="portada-autor">Isaac Martín de Diego</div>
            </div>
            
            <div class="portada-fecha">2025-2026</div>
        </div>
        
        <!-- Footer -->
        <div class="portada-footer">
            <img src="https://i.creativecommons.org/l/by-sa/4.0/88x31.png" 
                 alt="Licencia Creative Commons" 
                 class="portada-licencia-logo">
            
            <div class="portada-licencia-texto">
                Copyright © 2025 Víctor Aceña Gil, Isaac Martín de Diego. Esta obra está licenciada bajo CC BY-SA 4.0, Creative Commons Atribución-Compartir Igual 4.0 Internacional.
            </div>
        </div>
        
        <!-- Elementos decorativos -->
        <div class="portada-deco-lines">
            <div class="portada-deco-line"></div>
            <div class="portada-deco-line"></div>
            <div class="portada-deco-line"></div>
        </div>
    </div>
</div>

\renewcommand*\contentsname{Índice ejercicios}
{
\hypersetup{linkcolor=}
\setcounter{tocdepth}{2}
\tableofcontents
}

\bookmarksetup{startatroot}

\chapter{\texorpdfstring{\textbf{Introducción}}{Introducción}}\label{introducciuxf3n}

Este documento recopila una serie de ejercicios prácticos y teóricos
diseñados para complementar la asignatura ``Modelos Estadísticos de
Predicción'' del Grado en Matemáticas. El objetivo de esta colección es
afianzar los conocimientos adquiridos en cada tema, fomentando tanto la
comprensión de los fundamentos teóricos como la habilidad para
implementar y diagnosticar modelos en R.

\section{\texorpdfstring{\textbf{Estructura de los
Ejercicios}}{Estructura de los Ejercicios}}\label{estructura-de-los-ejercicios}

Los ejercicios están organizados por temas, siguiendo la estructura del
curso. Para cada tema, encontrarás una mezcla de:

\begin{itemize}
\tightlist
\item
  \textbf{Preguntas Conceptuales:} Diseñadas para reforzar la
  comprensión de la teoría subyacente.
\item
  \textbf{Problemas Prácticos con R:} Enfocados en la aplicación de las
  técnicas a conjuntos de datos reales o simulados.
\item
  \textbf{Ejercicios de Interpretación:} Centrados en la habilidad
  crítica de interpretar correctamente las salidas de los modelos
  estadísticos.
\end{itemize}

\section{\texorpdfstring{\textbf{Requisitos
Previos}}{Requisitos Previos}}\label{requisitos-previos}

Para abordar estos ejercicios, se asume que el estudiante ha estudiado
el contenido teórico del tema correspondiente y posee un manejo básico
del entorno de programación R y RStudio.

\bookmarksetup{startatroot}

\chapter{\texorpdfstring{\textbf{Regresión Lineal
Simple}}{Regresión Lineal Simple}}\label{regresiuxf3n-lineal-simple}

\subsection{Ejercicio 1: Fundamentos
Conceptuales}\label{ejercicio-1-fundamentos-conceptuales}

Basándote en el texto, explica con tus propias palabras por qué un
coeficiente de correlación de Pearson (\(r\)) alto no es suficiente para
modelar una relación y por qué la regresión lineal es un paso más allá.
Menciona al menos dos cosas que el modelo de regresión proporciona y que
la correlación por sí sola no ofrece.

\subsection{Ejercicio 2: Interpretación de
Coeficientes}\label{ejercicio-2-interpretaciuxf3n-de-coeficientes}

Un analista ajusta un modelo para predecir el gasto anual en compras
online (\texttt{gasto}, en euros) basándose en la edad del cliente
(\texttt{edad}). El modelo ajustado es:

\texttt{gasto\ =\ 1500\ +\ 12\ *\ edad}

\begin{enumerate}
\def\labelenumi{\alph{enumi})}
\tightlist
\item
  ¿Cuál es el gasto predicho para un cliente de 30 años?
\item
  Interpreta el significado de la pendiente (12) en el contexto
  específico de este problema.
\item
  Interpreta el significado del intercepto (1500). ¿Crees que esta
  interpretación tiene sentido práctico en el mundo real? ¿Por qué?
\end{enumerate}

\subsection{Ejercicio 3: Aplicación Práctica con R (Ajuste e
Inferencia)}\label{ejercicio-3-aplicaciuxf3n-pruxe1ctica-con-r-ajuste-e-inferencia}

Utiliza el conjunto de datos \texttt{pressure} de R, que contiene
mediciones de temperatura y presión de vapor de mercurio.

\begin{enumerate}
\def\labelenumi{\alph{enumi})}
\tightlist
\item
  Ajusta un modelo de regresión lineal simple para predecir la presión
  (\texttt{pressure}) en función de la temperatura
  (\texttt{temperature}). Guarda el modelo en un objeto.
\item
  Utiliza la función \texttt{summary()} sobre el objeto del modelo.
\item
  Interpreta el valor del \textbf{coeficiente de determinación R²}. ¿Qué
  porcentaje de la variabilidad de la presión es explicado por la
  temperatura?
\item
  Interpreta el \textbf{p-valor del estadístico F}. ¿Es el modelo útil
  en su conjunto?
\item
  ¿Es el coeficiente de la temperatura estadísticamente significativo a
  un nivel de \(\alpha = 0.05\)? Justifica tu respuesta basándote en el
  p-valor del test t.
\end{enumerate}

\subsection{Ejercicio 4: Intervalos de Confianza y
Predicción}\label{ejercicio-4-intervalos-de-confianza-y-predicciuxf3n}

Usando el modelo del ejercicio anterior
(\texttt{lm(pressure\ \textasciitilde{}\ temperature,\ data\ =\ pressure)}):

\begin{enumerate}
\def\labelenumi{\alph{enumi})}
\tightlist
\item
  Calcula el \textbf{intervalo de confianza al 95\%} para la
  \emph{presión media} esperada cuando la temperatura es de 250 grados.
\item
  Calcula el \textbf{intervalo de predicción al 95\%} para la presión de
  una \emph{única y nueva} medición realizada a 250 grados.
\item
  ¿Cuál de los dos intervalos es más ancho? Explica la razón teórica de
  esta diferencia.
\end{enumerate}

\subsection{Ejercicio 5: Supuestos del
Modelo}\label{ejercicio-5-supuestos-del-modelo}

Enumera los cuatro supuestos del modelo de regresión lineal clásico
(también conocidos como supuestos de Gauss-Markov) y explica brevemente
la importancia de cada uno.

\subsection{Ejercicio 6: Diagnóstico de Linealidad y
Homocedasticidad}\label{ejercicio-6-diagnuxf3stico-de-linealidad-y-homocedasticidad}

Para el modelo del ejercicio 3:

\begin{enumerate}
\def\labelenumi{\alph{enumi})}
\tightlist
\item
  Genera y muestra el gráfico de \textbf{Residuos vs.~Valores
  Ajustados}. Basándote en este gráfico, ¿se cumple el supuesto de
  \textbf{linealidad}? Explica en qué te basas.
\item
  Genera y muestra el gráfico \textbf{Scale-Location}. Basándote en este
  gráfico, ¿se cumple el supuesto de \textbf{homocedasticidad}? Describe
  el patrón que indicaría un problema de heterocedasticidad.
\end{enumerate}

\subsection{Ejercicio 7: Diagnóstico de
Normalidad}\label{ejercicio-7-diagnuxf3stico-de-normalidad}

Para el modelo del ejercicio 3:

\begin{enumerate}
\def\labelenumi{\alph{enumi})}
\tightlist
\item
  Genera un gráfico \textbf{Normal Q-Q} de los residuos. ¿Parecen seguir
  los residuos una distribución normal?
\item
  Realiza un \textbf{test de Shapiro-Wilk} sobre los residuos del
  modelo. ¿Qué concluyes a partir del p-valor?
\end{enumerate}

\subsection{Ejercicio 8: Descomposición de la Varianza
(ANOVA)}\label{ejercicio-8-descomposiciuxf3n-de-la-varianza-anova}

Explica qué representan la \textbf{Suma de Cuadrados Total (SST)}, la
\textbf{Suma de Cuadrados de la Regresión (SSR)} y la \textbf{Suma de
Cuadrados del Error (SSE)}. ¿Cuál es la ecuación fundamental que las
relaciona?

\subsection{Ejercicio 9: Observaciones
Influyentes}\label{ejercicio-9-observaciones-influyentes}

Basado en la teoría de los apuntes:

\begin{enumerate}
\def\labelenumi{\alph{enumi})}
\tightlist
\item
  Explica la diferencia entre un residuo simple (\(e_i\)), un residuo
  estandarizado y un residuo estudentizado. ¿Por qué se prefieren los
  estudentizados para el diagnóstico?
\item
  ¿Qué mide el \textbf{leverage} (\(h_{ii}\))? ¿Y la \textbf{distancia
  de Cook} (\(D_i\))? ¿Puede una observación tener un leverage alto y no
  ser influyente?
\end{enumerate}

\subsection{Ejercicio 10: Relación entre Pruebas de
Hipótesis}\label{ejercicio-10-relaciuxf3n-entre-pruebas-de-hipuxf3tesis}

En el contexto \textbf{exclusivo} de la regresión lineal simple, ¿qué
relación matemática existe entre el estadístico \textbf{F} del test
ANOVA y el estadístico \textbf{t} del test para la pendiente
\(\beta_1\)? ¿Qué implica esto para sus respectivos p-valores?

\bookmarksetup{startatroot}

\chapter{\texorpdfstring{\textbf{Regresión Lineal
Múltiple}}{Regresión Lineal Múltiple}}\label{regresiuxf3n-lineal-muxfaltiple}

\subsection{\texorpdfstring{Ejercicio 1: Conceptual (Interpretación
\emph{Ceteris
Paribus})}{Ejercicio 1: Conceptual (Interpretación Ceteris Paribus)}}\label{ejercicio-1-conceptual-interpretaciuxf3n-ceteris-paribus}

Un analista ajusta dos modelos para predecir el consumo de un coche
(\texttt{mpg}):

\begin{enumerate}
\def\labelenumi{\arabic{enumi}.}
\tightlist
\item
  \texttt{lm(mpg\ \textasciitilde{}\ wt)} obtiene un coeficiente para
  \texttt{wt} de -5.3.
\item
  \texttt{lm(mpg\ \textasciitilde{}\ wt\ +\ hp)} obtiene un coeficiente
  para \texttt{wt} de -3.8.
\end{enumerate}

Explica detalladamente por qué el coeficiente para la variable
\texttt{wt} (peso) cambia al añadir la variable \texttt{hp} (caballos de
fuerza). ¿Cuál de los dos coeficientes representa el efecto ``puro'' o
``aislado'' del peso? Fundamenta tu respuesta en el principio de
\textbf{\emph{ceteris paribus}}.

\subsection{Ejercicio 2: Práctico (Ajuste e Interpretación de un Modelo
Múltiple)}\label{ejercicio-2-pruxe1ctico-ajuste-e-interpretaciuxf3n-de-un-modelo-muxfaltiple}

Usa el conjunto de datos \texttt{iris} de R. Queremos modelar la anchura
del pétalo (\texttt{Petal.Width}) en función de la longitud del pétalo
(\texttt{Petal.Length}) y la anchura del sépalo (\texttt{Sepal.Width}).

\begin{enumerate}
\def\labelenumi{\alph{enumi})}
\tightlist
\item
  Ajusta un modelo de regresión lineal múltiple:
  \texttt{lm(Petal.Width\ \textasciitilde{}\ Petal.Length\ +\ Sepal.Width,\ data\ =\ iris)}.
\item
  Interpreta el coeficiente estimado para \texttt{Petal.Length}.
\item
  Interpreta el coeficiente estimado para \texttt{Sepal.Width}.
\item
  Interpreta el intercepto del modelo. ¿Tiene un significado práctico en
  este contexto biológico?
\end{enumerate}

\subsection{Ejercicio 3: Conceptual (R² vs.~R²
Ajustado)}\label{ejercicio-3-conceptual-ruxb2-vs.-ruxb2-ajustado}

Cuando pasamos de un modelo simple a uno múltiple, introducimos el
\textbf{R² ajustado} como medida de bondad de ajuste.

\begin{enumerate}
\def\labelenumi{\alph{enumi})}
\tightlist
\item
  ¿Cuál es el principal problema de usar el R² tradicional para comparar
  modelos con diferente número de predictores?
\item
  ¿Cómo soluciona el R² ajustado este problema? Explica qué
  ``penalización'' introduce en su fórmula.
\end{enumerate}

\subsection{Ejercicio 4: Interpretación de Salidas de
R}\label{ejercicio-4-interpretaciuxf3n-de-salidas-de-r}

Te presentan el siguiente resumen de un modelo que predice el prestigio
de una ocupación (\texttt{prestige}) en función de los ingresos
(\texttt{income}) y el nivel educativo (\texttt{education}).

\begin{verbatim}
Coefficients:
            Estimate Std. Error t value Pr(>|t|)    
(Intercept) -6.0647    4.2750   -1.419   0.1595    
income       0.0013    0.0003    4.524   1.9e-05 ***
education    4.1832    0.3887   10.762   < 2e-16 ***

Multiple R-squared:  0.79,  Adjusted R-squared:  0.785 
F-statistic: 185.6 on 2 and 99 DF,  p-value: < 2.2e-16
\end{verbatim}

\begin{enumerate}
\def\labelenumi{\alph{enumi})}
\tightlist
\item
  ¿Es el modelo globalmente significativo? ¿En qué te basas?
\item
  ¿Son los predictores \texttt{income} y \texttt{education}
  individualmente significativos, después de controlar por el efecto del
  otro? Justifica tu respuesta.
\item
  Explica la diferencia conceptual entre lo que evalúa el \textbf{test F
  global} y lo que evalúan los \textbf{tests t individuales} en este
  modelo.
\end{enumerate}

\subsection{Ejercicio 5: Conceptual
(Multicolinealidad)}\label{ejercicio-5-conceptual-multicolinealidad}

Describe con tus propias palabras qué es la \textbf{multicolinealidad}.
Menciona tres consecuencias negativas que puede tener la
multicolinealidad severa en un modelo de regresión y si afecta más a la
\textbf{predicción} o a la \textbf{inferencia}.

\subsection{Ejercicio 6: Práctico (Diagnóstico de
Multicolinealidad)}\label{ejercicio-6-pruxe1ctico-diagnuxf3stico-de-multicolinealidad}

Usa el dataset \texttt{mtcars}. Ajusta un modelo para predecir el
consumo (\texttt{mpg}) usando como predictores el número de cilindros
(\texttt{cyl}), la cilindrada (\texttt{disp}), los caballos de fuerza
(\texttt{hp}) y el peso (\texttt{wt}).

\begin{enumerate}
\def\labelenumi{\alph{enumi})}
\tightlist
\item
  Observa el \texttt{summary()} del modelo. ¿Hay alguna variable que, a
  pesar de tener una alta correlación simple con \texttt{mpg}, no
  resulte significativa en el modelo múltiple?
\item
  Carga la librería \texttt{car} y calcula el \textbf{Factor de
  Inflación de la Varianza (VIF)} para cada predictor.
\item
  Basándote en los valores del VIF, ¿qué variables presentan un problema
  de multicolinealidad? ¿Cuál es tu recomendación para simplificar el
  modelo?
\end{enumerate}

\subsection{Ejercicio 7: Teórico (Notación
Matricial)}\label{ejercicio-7-teuxf3rico-notaciuxf3n-matricial}

\begin{enumerate}
\def\labelenumi{\alph{enumi})}
\tightlist
\item
  Escribe la fórmula del estimador de Mínimos Cuadrados Ordinarios
  (\(\hat{\mathbf{\beta}}\)) en notación matricial.
\item
  ¿Qué supuesto fundamental del modelo de regresión múltiple garantiza
  que la matriz \((\mathbf{X}^T\mathbf{X})\) sea invertible?
\end{enumerate}

\subsection{Ejercicio 8: Práctico (Gráficos de Regresión
Parcial)}\label{ejercicio-8-pruxe1ctico-gruxe1ficos-de-regresiuxf3n-parcial}

Usa el dataset \texttt{Prestige} de la librería \texttt{car}.

\begin{enumerate}
\def\labelenumi{\alph{enumi})}
\tightlist
\item
  Ajusta el modelo
  \texttt{lm(prestige\ \textasciitilde{}\ income\ +\ education\ +\ women,\ data\ =\ Prestige)}.
\item
  Genera los gráficos de regresión parcial (o ``added-variable plots'')
  para este modelo usando la función \texttt{avPlots(tu\_modelo)}.
\item
  Explica qué representa el gráfico para la variable \texttt{education}.
  ¿Qué significan los ejes X e Y de ese gráfico específico? ¿A qué
  corresponde la pendiente de la línea en ese gráfico?
\end{enumerate}

\subsection{Ejercicio 9: Inferencia (F-test
vs.~t-tests)}\label{ejercicio-9-inferencia-f-test-vs.-t-tests}

Describe un escenario hipotético en el que el \textbf{test F global} de
un modelo de regresión múltiple sea altamente significativo (p
\textless{} 0.001), pero \textbf{ninguno de los tests t individuales}
para los coeficientes sea significativo. ¿Cuál es la causa estadística
más probable de este fenómeno?

\subsection{Ejercicio 10: Práctico (Comparación de Modelos
Anidados)}\label{ejercicio-10-pruxe1ctico-comparaciuxf3n-de-modelos-anidados}

Usa el dataset \texttt{swiss}.

\begin{enumerate}
\def\labelenumi{\alph{enumi})}
\tightlist
\item
  Ajusta un \textbf{modelo reducido} para predecir \texttt{Fertility}
  usando solo \texttt{Agriculture} y \texttt{Education}.
\item
  Ajusta un \textbf{modelo completo} que, además de las variables
  anteriores, incluya \texttt{Catholic} y \texttt{Infant.Mortality}.
\item
  Utiliza la función \texttt{anova()} para comparar formalmente los dos
  modelos. ¿Aportan las variables \texttt{Catholic} y
  \texttt{Infant.Mortality} una mejora estadísticamente significativa al
  modelo? Interpreta el p-valor del test F resultante.
\end{enumerate}

\bookmarksetup{startatroot}

\chapter{\texorpdfstring{\textbf{Ingeniería de
Características}}{Ingeniería de Características}}\label{ingenieruxeda-de-caracteruxedsticas}

\subsection{Ejercicio 1: Conceptual (Diagnóstico antes de
Transformar)}\label{ejercicio-1-conceptual-diagnuxf3stico-antes-de-transformar}

El texto desaconseja fuertemente el enfoque de ``ensayo y error'' al
aplicar transformaciones. Explica con tus propias palabras por qué la
práctica de probar transformaciones hasta que mejore el R² es
metodológicamente peligrosa. Menciona al menos tres de los riesgos
específicos discutidos en los apuntes.

\subsection{Ejercicio 2: Práctico (Escalado de
Variables)}\label{ejercicio-2-pruxe1ctico-escalado-de-variables}

Utiliza el dataset \texttt{iris} de R y céntrate en las cuatro variables
predictoras continuas (\texttt{Sepal.Length}, \texttt{Sepal.Width},
\texttt{Petal.Length}, \texttt{Petal.Width}).

\begin{enumerate}
\def\labelenumi{\alph{enumi})}
\tightlist
\item
  Calcula la media y la desviación estándar de estas cuatro variables en
  su escala original. ¿Son sus escalas directamente comparables?
\item
  Crea un nuevo data frame donde hayas aplicado la
  \textbf{estandarización Z-Score} a estas cuatro variables. Verifica
  que las nuevas variables tienen una media cercana a 0 y una desviación
  estándar de 1.
\item
  ¿Por qué este paso de escalado es crucial antes de aplicar métodos de
  regularización como Ridge o Lasso, tal y como se menciona en el texto?
\end{enumerate}

\subsection{Ejercicio 3: Conceptual (Elección del Método de
Escalado)}\label{ejercicio-3-conceptual-elecciuxf3n-del-muxe9todo-de-escalado}

Describe un escenario hipotético para cada uno de los siguientes casos,
explicando por qué el método de escalado elegido sería el más apropiado:

\begin{enumerate}
\def\labelenumi{\alph{enumi})}
\tightlist
\item
  Un escenario donde la \textbf{estandarización Z-Score} es preferible.
\item
  Un escenario donde la \textbf{normalización Min-Max} es preferible.
\item
  Un escenario donde el \textbf{escalado robusto} (usando mediana y IQR)
  es necesario.
\end{enumerate}

\subsection{Ejercicio 4: Práctico (Transformación para
Linealizar)}\label{ejercicio-4-pruxe1ctico-transformaciuxf3n-para-linealizar}

En el tema anterior vimos que la relación en el dataset \texttt{cars}
(entre \texttt{speed} y \texttt{dist}) no era perfectamente lineal.

\begin{enumerate}
\def\labelenumi{\alph{enumi})}
\tightlist
\item
  Ajusta el modelo
  \texttt{lm(dist\ \textasciitilde{}\ speed,\ data\ =\ cars)} y genera
  el gráfico de residuos vs.~valores ajustados para confirmar
  visualmente la no linealidad (patrón curvo).
\item
  Los apuntes sugieren que la transformación logarítmica es útil para
  relaciones con ``rendimientos decrecientes''. Propón y aplica una
  transformación (ej. sobre el predictor, la respuesta, o ambos) para
  intentar linealizar la relación. Por ejemplo, ajusta
  \texttt{lm(log(dist)\ \textasciitilde{}\ speed,\ data\ =\ cars)}.
\item
  Genera de nuevo el gráfico de residuos vs.~valores ajustados para el
  nuevo modelo. Compara ambos diagnósticos. ¿Ha mejorado la linealidad?
\end{enumerate}

\subsection{Ejercicio 5: Práctico (Transformación de
Box-Cox)}\label{ejercicio-5-pruxe1ctico-transformaciuxf3n-de-box-cox}

Usa el dataset \texttt{Boston} de la librería \texttt{MASS}. La variable
respuesta \texttt{medv} (valor mediano de la vivienda) es estrictamente
positiva y tiene cierta asimetría.

\begin{enumerate}
\def\labelenumi{\alph{enumi})}
\tightlist
\item
  Carga la librería \texttt{MASS} y utiliza la función \texttt{boxcox()}
  para encontrar el valor de \(\lambda\) óptimo para la variable
  \texttt{medv} en un modelo simple frente a \texttt{lstat}. La fórmula
  sería
  \texttt{boxcox(medv\ \textasciitilde{}\ lstat,\ data\ =\ Boston)}.
\item
  Observando el gráfico que se genera, ¿a qué valor ``simple'' (como -1,
  0, 0.5, 1) se aproxima el \(\lambda\) óptimo?
\item
  Basándote en este resultado, ¿cuál de las transformaciones clásicas
  (logarítmica, raíz cuadrada, inversa, etc.) sería la más recomendable
  para la variable \texttt{medv}?
\end{enumerate}

\subsection{Ejercicio 6: Conceptual (Codificación de Variables
Categóricas)}\label{ejercicio-6-conceptual-codificaciuxf3n-de-variables-categuxf3ricas}

Explica la diferencia fundamental entre la \textbf{Codificación Ordinal}
y la \textbf{Codificación One-Hot}. Para cada una de las siguientes
variables, indica qué método de codificación usarías y justifica tu
elección:

\begin{itemize}
\tightlist
\item
  \texttt{mes}: (``Enero'', ``Febrero'', ``Marzo'', \ldots)
\item
  \texttt{nivel\_riesgo}: (``Bajo'', ``Medio'', ``Alto'', ``Crítico'')
\item
  \texttt{pais\_origen}: (``España'', ``Francia'', ``Alemania'',
  ``Italia'')
\end{itemize}

\subsection{Ejercicio 7: Práctico (Interacción entre Variables
Continuas)}\label{ejercicio-7-pruxe1ctico-interacciuxf3n-entre-variables-continuas}

Usa el dataset \texttt{mtcars} para investigar si el efecto del peso de
un coche (\texttt{wt}) sobre su consumo (\texttt{mpg}) depende de su
potencia (\texttt{hp}).

\begin{enumerate}
\def\labelenumi{\alph{enumi})}
\tightlist
\item
  Ajusta un modelo que incluya un término de interacción entre
  \texttt{wt} y \texttt{hp}. Escribe la fórmula en R.
\item
  Observa el \texttt{summary()} del modelo. ¿Es el término de
  interacción (\texttt{wt:hp}) estadísticamente significativo a un nivel
  de \(\alpha = 0.05\)?
\item
  Basándote en el signo del coeficiente de la interacción, ¿cómo cambia
  el efecto del peso sobre el consumo a medida que aumenta la potencia?
  (Es decir, ¿el efecto negativo del peso se hace más fuerte o más débil
  en los coches más potentes?).
\end{enumerate}

\subsection{Ejercicio 8: Interpretación de una Interacción (Continua x
Categórica)}\label{ejercicio-8-interpretaciuxf3n-de-una-interacciuxf3n-continua-x-categuxf3rica}

Un investigador modela el salario (\texttt{salario}, en euros) en
función de los años de experiencia (\texttt{experiencia}) y si el
empleado tiene o no un máster (\texttt{master}, con ``No'' como
categoría de referencia). El modelo ajustado es:

\texttt{salario\ =\ 30000\ +\ 1200*experiencia\ +\ 8000*masterSi\ +\ 300*experiencia:masterSi}

\begin{enumerate}
\def\labelenumi{\alph{enumi})}
\tightlist
\item
  Escribe la ecuación de regresión específica para los empleados que
  \textbf{no tienen} un máster.
\item
  Escribe la ecuación de regresión específica para los empleados que
  \textbf{sí tienen} un máster.
\item
  Interpreta el coeficiente de la interacción (300). ¿Qué nos dice sobre
  el retorno económico de la experiencia para ambos grupos?
\end{enumerate}

\subsection{Ejercicio 9: Conceptual (Principio de
Jerarquía)}\label{ejercicio-9-conceptual-principio-de-jerarquuxeda}

Explica el \textbf{principio de jerarquía} en el contexto de los modelos
de regresión con interacciones. Si un modelo incluye el término de
interacción \texttt{A:B}, ¿por qué es una buena práctica incluir siempre
los efectos principales \texttt{A} y \texttt{B}, incluso si sus tests t
individuales no son significativos?

\subsection{Ejercicio 10: Conceptual (Ingeniería de Características
Avanzada)}\label{ejercicio-10-conceptual-ingenieruxeda-de-caracteruxedsticas-avanzada}

Los apuntes discuten la creación de nuevas variables mediante
\textbf{ratios} y \textbf{combinaciones}. Para cada uno de los
siguientes escenarios, propón una nueva variable (feature) que podrías
crear y explica qué relación podría capturar mejor que las variables
originales por sí solas.

\begin{enumerate}
\def\labelenumi{\alph{enumi})}
\tightlist
\item
  Para predecir la rentabilidad de una tienda, tienes las variables
  \texttt{ventas\_totales} y \texttt{numero\_de\_empleados}.
\item
  Para predecir el riesgo de impago de un solicitante de préstamo,
  tienes las variables \texttt{ingresos\_anuales} y
  \texttt{deuda\_total}.
\end{enumerate}

\bookmarksetup{startatroot}

\chapter{\texorpdfstring{\textbf{Selección de variables, Regularización
y
Validación}}{Selección de variables, Regularización y Validación}}\label{selecciuxf3n-de-variables-regularizaciuxf3n-y-validaciuxf3n}

\subsection{Ejercicio 1: Conceptual (Sobreajuste
vs.~Subajuste)}\label{ejercicio-1-conceptual-sobreajuste-vs.-subajuste}

Explica con tus propias palabras qué es el \textbf{sobreajuste
(overfitting)} y el \textbf{subajuste (underfitting)}. Describe los
síntomas de cada uno comparando el error de entrenamiento con el error
de validación (o de test), y menciona la solución principal para cada
problema.

\subsection{Ejercicio 2: Práctico (Filtrado
Básico)}\label{ejercicio-2-pruxe1ctico-filtrado-buxe1sico}

Imagina que recibes un nuevo conjunto de datos con 50 predictores para
un modelo de regresión. Antes de aplicar métodos computacionalmente
costosos, decides hacer un filtrado inicial. Describe los \textbf{cuatro
criterios básicos} que aplicarías para descartar variables de forma
preliminar, según lo explicado en los apuntes.

\subsection{Ejercicio 3: Conceptual (AIC
vs.~BIC)}\label{ejercicio-3-conceptual-aic-vs.-bic}

Tanto el AIC como el BIC son criterios para comparar modelos, pero se
basan en filosofías distintas y tienen penalizaciones diferentes.

\begin{enumerate}
\def\labelenumi{\alph{enumi})}
\tightlist
\item
  Escribe la fórmula de la penalización por complejidad para el AIC y
  para el BIC.
\item
  ¿Cuál de los dos criterios tenderá a seleccionar modelos más simples
  (más parsimoniosos)? ¿Por qué?
\item
  Si tu objetivo principal es la \textbf{precisión predictiva}, ¿cuál de
  los dos criterios es generalmente preferido?
\end{enumerate}

\subsection{Ejercicio 4: Práctico (Best Subset y Criterios de
Información)}\label{ejercicio-4-pruxe1ctico-best-subset-y-criterios-de-informaciuxf3n}

Usa el conjunto de datos \texttt{mtcars} y la librería \texttt{leaps}.

\begin{enumerate}
\def\labelenumi{\alph{enumi})}
\tightlist
\item
  Utiliza la función \texttt{regsubsets()} para realizar una selección
  del mejor subconjunto (\texttt{best\ subset\ selection}) para predecir
  \texttt{mpg} usando el resto de variables.
\item
  Obtén el \texttt{summary()} de los resultados. ¿Qué modelo (cuántas
  variables) es el mejor según el criterio \textbf{Cp de Mallows}?
\item
  ¿Y cuál es el mejor modelo según el \textbf{R² ajustado}?
\item
  ¿Coinciden ambos criterios en el número de variables del modelo
  óptimo?
\end{enumerate}

\subsection{Ejercicio 5: Conceptual (Métodos
Stepwise)}\label{ejercicio-5-conceptual-muxe9todos-stepwise}

Los métodos automáticos paso a paso (forward, backward, stepwise) son
computacionalmente eficientes, pero el texto advierte sobre su uso.
Menciona y explica brevemente \textbf{tres de las principales
limitaciones o problemas} de estos métodos.

\subsection{Ejercicio 6: Práctico (Selección Backward
Stepwise)}\label{ejercicio-6-pruxe1ctico-selecciuxf3n-backward-stepwise}

Utiliza el conjunto de datos \texttt{swiss} para predecir
\texttt{Fertility}.

\begin{enumerate}
\def\labelenumi{\alph{enumi})}
\tightlist
\item
  Ajusta el modelo completo:
  \texttt{modelo\_completo\ \textless{}-\ lm(Fertility\ \textasciitilde{}\ .,\ data\ =\ swiss)}.
\item
  Utiliza la función \texttt{step()} para realizar una selección
  \textbf{regresiva (backward)} basada en el criterio AIC.
\item
  Reporta la fórmula del modelo final que selecciona el algoritmo y su
  valor de AIC.
\end{enumerate}

\subsection{Ejercicio 7: Conceptual (Ridge
vs.~Lasso)}\label{ejercicio-7-conceptual-ridge-vs.-lasso}

La regresión Ridge y Lasso son dos métodos de regularización muy
populares, pero tienen un efecto fundamentalmente diferente sobre los
coeficientes del modelo.

\begin{enumerate}
\def\labelenumi{\alph{enumi})}
\tightlist
\item
  ¿Qué tipo de penalización utiliza cada método (\(L_1\) o \(L_2\))?
\item
  ¿Cuál de los dos métodos puede realizar selección de variables (es
  decir, anular coeficientes por completo)?
\item
  Describe un escenario en el que preferirías usar Ridge sobre Lasso.
\end{enumerate}

\subsection{Ejercicio 8: Práctico (Regresión
Lasso)}\label{ejercicio-8-pruxe1ctico-regresiuxf3n-lasso}

Utiliza el paquete \texttt{glmnet} y el conjunto de datos
\texttt{mtcars} para predecir \texttt{mpg}.

\begin{enumerate}
\def\labelenumi{\alph{enumi})}
\tightlist
\item
  Prepara los datos: crea una matriz \texttt{x} para los predictores y
  un vector \texttt{y} para la respuesta.
\item
  Utiliza la función \texttt{cv.glmnet()} para realizar una validación
  cruzada y encontrar el valor de \texttt{lambda} óptimo para una
  regresión \textbf{Lasso} (\texttt{alpha\ =\ 1}).
\item
  Extrae y muestra los coeficientes del modelo Lasso ajustado con el
  \texttt{lambda.min}.
\item
  ¿Qué variables ha eliminado el modelo (coeficientes iguales a cero)?
\end{enumerate}

\subsection{Ejercicio 9: Conceptual
(Validación)}\label{ejercicio-9-conceptual-validaciuxf3n}

Explica la diferencia entre la estrategia de validación
\textbf{Train/Test Split simple} y la \textbf{Validación Cruzada
k-fold}. ¿Cuál es la principal ventaja de la validación cruzada sobre la
división simple? ¿En qué situación (tamaño del dataset) recomendarías
usar cada una?

\subsection{Ejercicio 10: Práctico (Validación
Cruzada)}\label{ejercicio-10-pruxe1ctico-validaciuxf3n-cruzada}

Imagina que has ajustado dos modelos para predecir \texttt{mpg} en el
dataset \texttt{mtcars}: 1. Un modelo simple:
\texttt{mpg\ \textasciitilde{}\ wt\ +\ hp} 2. Un modelo complejo:
\texttt{mpg\ \textasciitilde{}\ .} (todas las variables)

Utilizando la librería \texttt{caret} y la función \texttt{train()},
como se muestra en el \texttt{callout-tip} ``La maldición del
sobreajuste'', configura y ejecuta una \textbf{validación cruzada de 10
particiones} para estimar el \textbf{RMSE} de ambos modelos. ¿Cuál de
los dos modelos generaliza mejor a nuevos datos según esta estimación?

\bookmarksetup{startatroot}

\chapter{\texorpdfstring{\textbf{Modelos de Regresión
Generalizada}}{Modelos de Regresión Generalizada}}\label{modelos-de-regresiuxf3n-generalizada}

\subsection{Ejercicio 1: Conceptual (Fundamentos de
GLM)}\label{ejercicio-1-conceptual-fundamentos-de-glm}

Explica los \textbf{tres componentes clave} que definen a cualquier
Modelo Lineal Generalizado (GLM) y describe brevemente la función de
cada uno.

\subsection{Ejercicio 2: Conceptual (Función de
Enlace)}\label{ejercicio-2-conceptual-funciuxf3n-de-enlace}

¿Cuál es el propósito fundamental de la \textbf{función de enlace} en un
GLM? ¿Por qué la regresión lineal clásica es considerada un caso
particular de un GLM? (Pista: piensa en su función de enlace).

\subsection{Ejercicio 3: Práctico (Ajuste de un Modelo
Logístico)}\label{ejercicio-3-pruxe1ctico-ajuste-de-un-modelo-loguxedstico}

Usa el conjunto de datos \texttt{mtcars} de R. La variable \texttt{am}
indica si la transmisión de un coche es automática (0) o manual (1).

\begin{enumerate}
\def\labelenumi{\alph{enumi})}
\tightlist
\item
  Ajusta un modelo de regresión logística para predecir la probabilidad
  de que una transmisión sea manual (\texttt{am}) en función del peso
  del coche (\texttt{wt}) y los caballos de fuerza (\texttt{hp}).
\item
  Utiliza la función \texttt{summary()} para examinar el modelo. ¿Qué
  variables parecen ser significativas?
\item
  Obtén los coeficientes del modelo. ¿Cómo interpretarías el signo del
  coeficiente para la variable \texttt{wt}?
\end{enumerate}

\subsection{Ejercicio 4: Interpretación (Odds
Ratios)}\label{ejercicio-4-interpretaciuxf3n-odds-ratios}

Basado en el modelo del ejercicio anterior:

\begin{enumerate}
\def\labelenumi{\alph{enumi})}
\tightlist
\item
  Calcula el \textbf{Odds Ratio (OR)} para el coeficiente de la variable
  \texttt{hp}.
\item
  Interpreta este Odds Ratio en el contexto del problema.
  Específicamente, ¿cómo cambian las ``odds'' (la razón de probabilidad)
  de tener una transmisión manual por cada caballo de fuerza adicional,
  manteniendo el peso constante?
\end{enumerate}

\subsection{Ejercicio 5: Práctico (Validación del Modelo
Logístico)}\label{ejercicio-5-pruxe1ctico-validaciuxf3n-del-modelo-loguxedstico}

Continuando con el modelo logístico de \texttt{mtcars}:

\begin{enumerate}
\def\labelenumi{\alph{enumi})}
\tightlist
\item
  Genera las predicciones de probabilidad del modelo para los datos.
\item
  Convierte estas probabilidades en clases (``0'' o ``1'') usando un
  umbral de decisión de 0.5.
\item
  Crea la \textbf{matriz de confusión} comparando las predicciones con
  los valores reales.
\item
  Calcula la \textbf{precisión (accuracy)} global del modelo.
\item
  (Bonus) Utiliza el paquete \texttt{pROC} para calcular y visualizar la
  \textbf{curva ROC} y obtener el valor del \textbf{AUC}. ¿Qué tan buena
  es la capacidad discriminativa del modelo?
\end{enumerate}

\subsection{Ejercicio 6: Conceptual (Regresión de
Poisson)}\label{ejercicio-6-conceptual-regresiuxf3n-de-poisson}

\begin{enumerate}
\def\labelenumi{\alph{enumi})}
\tightlist
\item
  ¿Qué tipo de variable respuesta está diseñada para modelar la
  regresión de Poisson?
\item
  ¿Cuál es el supuesto fundamental de la distribución de Poisson
  respecto a la relación entre la media y la varianza?
\item
  ¿Cómo se llama el problema que surge cuando este supuesto se viola y
  la varianza es mayor que la media?
\end{enumerate}

\subsection{Ejercicio 7: Práctico (Ajuste de un Modelo de
Poisson)}\label{ejercicio-7-pruxe1ctico-ajuste-de-un-modelo-de-poisson}

El dataset \texttt{discoveries} de R es una serie temporal que cuenta el
número de ``grandes inventos'' por año.

\begin{enumerate}
\def\labelenumi{\alph{enumi})}
\tightlist
\item
  Crea un gráfico de la serie temporal. ¿Parece la media del conteo
  constante a lo largo del tiempo?
\item
  Ajusta un modelo de regresión de Poisson simple donde
  \texttt{discoveries} es la respuesta y el tiempo
  (\texttt{time(discoveries)}) es el predictor.
\item
  Interpreta el coeficiente del tiempo. (Pista: recuerda exponenciarlo
  para obtener el Incidence Rate Ratio - IRR).
\end{enumerate}

\subsection{Ejercicio 8: Diagnóstico
(Sobredispersión)}\label{ejercicio-8-diagnuxf3stico-sobredispersiuxf3n}

\begin{enumerate}
\def\labelenumi{\alph{enumi})}
\tightlist
\item
  Para el modelo de Poisson del ejercicio anterior, calcula el
  \textbf{estadístico de dispersión (\(\hat{\phi}\))}. (Pista:
  \(\hat{\phi} = \frac{\sum r_i^2}{n-p}\), donde los \(r_i\) son los
  residuos Pearson).
\item
  Basándote en el valor de \(\hat{\phi}\), ¿hay evidencia de
  sobredispersión?
\item
  Si encuentras sobredispersión, ¿cuál es el modelo alternativo que
  proponen los apuntes? ¿Qué ventaja teórica ofrece este modelo
  alternativo?
\end{enumerate}

\subsection{Ejercicio 9: Conceptual
(Deviance)}\label{ejercicio-9-conceptual-deviance}

La \textbf{deviance} es la medida principal de bondad de ajuste en los
GLM. Explica conceptualmente qué mide. ¿Cómo se utiliza la diferencia en
deviance entre dos modelos anidados para decidir cuál es mejor?

\subsection{Ejercicio 10: Elección del Modelo
Adecuado}\label{ejercicio-10-elecciuxf3n-del-modelo-adecuado}

Para cada uno de los siguientes escenarios, indica qué tipo de GLM
(Logístico, Poisson, Binomial Negativo, Gamma\ldots) sería el más
apropiado y por qué.

\begin{enumerate}
\def\labelenumi{\alph{enumi})}
\tightlist
\item
  Quieres modelar el \textbf{tiempo (en minutos)} que tarda un cliente
  en resolver una consulta en un centro de atención telefónica. El
  tiempo es siempre positivo y muchos valores se agrupan en tiempos
  cortos, con una cola larga de tiempos muy largos.
\item
  Quieres predecir la \textbf{presencia o ausencia} de una especie de
  planta en diferentes parcelas de un bosque.
\item
  Quieres modelar el \textbf{número de visitas} que cada usuario hace a
  una página web en un mes. Observas que la varianza del número de
  visitas es mucho mayor que la media.
\end{enumerate}

\bookmarksetup{startatroot}

\chapter{\texorpdfstring{\textbf{Ejercicios
Avanzados}}{Ejercicios Avanzados}}\label{ejercicios-avanzados}

\subsection{Ejercicio 1: Derivación de
Estimadores}\label{ejercicio-1-derivaciuxf3n-de-estimadores}

Considera el modelo de regresión lineal simple
\(Y_i = \beta_0 + \beta_1 X_i + \varepsilon_i\). Partiendo de la función
objetivo de Mínimos Cuadrados Ordinarios (MCO),
\(S(\beta_0, \beta_1) = \sum_{i=1}^{n} (y_i - \beta_0 - \beta_1 x_i)^2\),
realiza la derivación matemática completa para obtener las expresiones
de los estimadores \(\hat{\beta}_0\) y \(\hat{\beta}_1\). Muestra todos
los pasos, desde el cálculo de las derivadas parciales hasta la
resolución de las ecuaciones normales.

\subsection{Ejercicio 2: El Impacto de la
Multicolinealidad}\label{ejercicio-2-el-impacto-de-la-multicolinealidad}

En un modelo de regresión múltiple con dos predictores estandarizados
(\(X_1, X_2\)), la varianza del estimador \(\hat{\beta}_1\) viene dada
por \(\text{Var}(\hat{\beta}_1) = \frac{\sigma^2}{n(1-r_{12}^2)}\),
donde \(r_{12}\) es la correlación entre \(X_1\) y \(X_2\).

\begin{enumerate}
\def\labelenumi{\alph{enumi})}
\tightlist
\item
  Explica matemáticamente qué le ocurre a la varianza de
  \(\hat{\beta}_1\) cuando la correlación entre los predictores
  (\(r_{12}\)) se aproxima a 1 (multicolinealidad perfecta).
\item
  Relaciona esta fórmula con la del Factor de Inflación de la Varianza
  (VIF). ¿Cómo demuestra esta expresión que la multicolinealidad
  ``infla'' la varianza de los estimadores de los coeficientes?
\end{enumerate}

\subsection{Ejercicio 3: Interpretación de Coeficientes en Modelos
Transformados}\label{ejercicio-3-interpretaciuxf3n-de-coeficientes-en-modelos-transformados}

Considera un modelo de regresión \textbf{log-log}:
\(\log(Y_i) = \beta_0 + \beta_1 \log(X_i) + \varepsilon_i\). Demuestra
matemáticamente que el coeficiente \(\beta_1\) puede interpretarse como
una \textbf{elasticidad}, es decir, el cambio porcentual en \(Y\) ante
un cambio del 1\% en \(X\). (Pista: utiliza la derivada de \(\log(Y)\)
con respecto a \(\log(X)\)).

\subsection{Ejercicio 4: Fundamentos de la
Regularización}\label{ejercicio-4-fundamentos-de-la-regularizaciuxf3n}

Explica desde una perspectiva geométrica por qué la regularización
\textbf{Lasso (penalización L1)} es capaz de reducir los coeficientes
exactamente a cero, realizando así selección de variables, mientras que
la regularización \textbf{Ridge (penalización L2)} solo puede encoger
los coeficientes hacia cero sin anularlos por completo. Apoya tu
explicación con un dibujo o descripción de las ``regiones de
restricción'' de ambos métodos en un espacio de dos coeficientes
(\(\beta_1, \beta_2\)).

\subsection{Ejercicio 5: La Familia Exponencial y los
GLM}\label{ejercicio-5-la-familia-exponencial-y-los-glm}

La teoría de los Modelos Lineales Generalizados (GLM) se basa en que
distribuciones como la Normal, Binomial o Poisson pertenecen a la
\textbf{familia exponencial}. La forma canónica de esta familia
establece una relación directa entre la media y la varianza a través de
la \textbf{función de varianza} \(V(\mu)\). Explica cuál es la función
de varianza para un modelo de \textbf{Poisson} y para un modelo
\textbf{Binomial}. ¿Qué implicaciones tiene la forma de \(V(\mu)\) en
cada caso sobre el comportamiento de los datos y los supuestos del
modelo?

\subsection{Ejercicio 6: El Problema de la Inferencia en Métodos
Stepwise}\label{ejercicio-6-el-problema-de-la-inferencia-en-muxe9todos-stepwise}

Los apuntes advierten que los p-valores de un modelo final obtenido
mediante selección por pasos (stepwise) están \textbf{sesgados y son
excesivamente optimistas}. Explica el razonamiento estadístico detrás de
esta advertencia. ¿Por qué el proceso iterativo de ``buscar y
seleccionar'' la variable más significativa en cada paso invalida los
supuestos teóricos del test t estándar?

\subsection{Ejercicio 7: Propiedades de los Estimadores
MCO}\label{ejercicio-7-propiedades-de-los-estimadores-mco}

El \textbf{Teorema de Gauss-Markov} establece que, bajo ciertos
supuestos, los estimadores de Mínimos Cuadrados Ordinarios (MCO) son
\textbf{MELI (Mejores Estimadores Lineales Insesgados)}. Demuestra la
propiedad de \textbf{insesgadez} para el estimador
\(\hat{\boldsymbol{\beta}}\) en notación matricial. Es decir, demuestra
que \(E[\hat{\boldsymbol{\beta}}] = \boldsymbol{\beta}\). Muestra todos
los pasos y menciona qué supuestos del modelo estás utilizando en cada
paso.

\subsection{Ejercicio 8: Intervalos de Confianza
vs.~Predicción}\label{ejercicio-8-intervalos-de-confianza-vs.-predicciuxf3n}

La fórmula para el intervalo de predicción para una nueva observación en
regresión lineal simple es:
\[\hat{y}_0 \pm t_{\alpha/2, n-2} \cdot \sqrt{\text{MSE} \left( 1 + \frac{1}{n} + \frac{(x_0 - \bar{x})^2}{S_{xx}} \right)}\]
Explica el origen y el significado de cada uno de los \textbf{tres
términos} que se encuentran dentro del paréntesis bajo la raíz cuadrada.
¿Qué fuente de incertidumbre representa cada término y por qué la suma
de los tres es necesaria para un intervalo de predicción?

\subsection{Ejercicio 9: Estimación por Máxima
Verosimilitud}\label{ejercicio-9-estimaciuxf3n-por-muxe1xima-verosimilitud}

Para un modelo de regresión logística, la función de log-verosimilitud
es:
\[\ell(\boldsymbol{\beta}) = \sum_{i=1}^{n} \left[y_i \log(p_i) + (1-y_i) \log(1-p_i)\right]\]
donde \(p_i = \frac{1}{1 + e^{-\mathbf{x}_i^T\boldsymbol{\beta}}}\).
Deriva la \textbf{ecuación de puntuación (score equation)} para un
coeficiente \(\beta_j\) (es decir, calcula
\(\frac{\partial \ell}{\partial \beta_j}\)) y demuestra que se iguala a
cero cuando \(\sum_{i=1}^{n} x_{ij}(y_i - p_i) = 0\). Interpreta el
significado de esta condición final.

\subsection{Ejercicio 10: El Coeficiente de Regresión
Parcial}\label{ejercicio-10-el-coeficiente-de-regresiuxf3n-parcial}

El texto afirma que el coeficiente \(\hat{\beta}_j\) de una regresión
múltiple puede entenderse como el coeficiente de una regresión simple
entre dos conjuntos de residuos. Explica con detalle este concepto de
\textbf{regresión parcial}. ¿Qué se está ``parcializando'' o
``eliminando'' de la variable respuesta \(Y\) y del predictor \(X_j\)
antes de calcular su relación? ¿Por qué este concepto es fundamental
para entender la interpretación \emph{ceteris paribus}?




\end{document}
